\documentclass[12pt, oneside]{article}
%packages
\usepackage{geometry}                		                  		
\usepackage{graphicx}				
\usepackage{xcolor}
\usepackage{graphicx}
\usepackage{tcolorbox}					
\usepackage{amssymb}
\PassOptionsToPackage{hyphens}{url}\usepackage{hyperref}
\usepackage[utf8]{inputenc}
%\usepackage[T1]{fontenc}
\usepackage[naustrian]{babel}
\usepackage{amsmath}
\usepackage{minted}
%layout
\geometry{letterpaper} 

\title{
Übung \\ \vspace{0.3cm}
\Large{Lambdas} 
}
\begin{document}
 \setlength{\parindent}{0em} 
\maketitle
\textcolor{red}{Besprechung am 4.April 2020}
\section{Die Aufgabe}
Programmiere ein funktionales Interface mit dem Namen NumberSelector, dessen Methode \textit{check} einen int-Wert als Parameter hat und boolean returnt. \\ \\
	Implementiere in NumberSelector eine statische Methode \textit{findFirst}, die als \\ Parameter einen NumberSelector und eine int-Zahl hat. Die Methode soll die ersten \\ $n$ int-Werte $n > 0$ und $n < 1.000.000$ finden, für die der Selector true zurückgibt und diese als int[]-Array zurückgeben. \\ \\
	\textbf{Folgendes soll gefunden werden:}
	\begin{itemize}
		\item die ersten 10 Zahlen, die durch 7 teilbar sind
		\item die ersten 10 Primzahlen \\$\left(p_{n}\right)_{n \in \mathbb{N}}=(2,3,5,7,11,13,17,19,23,29,\ldots)$
		\item die Zahlen von 0 bis 20
	\end{itemize}
\vspace{0.5cm}	
\begin{tcolorbox}
		Ein functional Interface gibt es in Java seit der Version 8 (seit den Lambdas).
		Dieses Interface besitzt nur \textbf{eine} abstrakte-Methode \textbf{nicht} \\ implementierte Methode. Dieses Interface kann noch zusätzliche statische \\ implementierte und default-Methoden besitzen. Sie besitzen außerdem die \\ Annotation @FunctionalInterface. \\ Solche Schnittstellen werden von üblicherweise von Lambda-Expressions \\ implementiert.
\end{tcolorbox}

\subsection*{Ein Beispiel}

\begin{listing}[ht]
\begin{minted}[frame=single, style=vim, bgcolor=black]{java}
@FunctionalInterface
public interface MyFunctionalInterface {
   void execute();
}
\end{minted}
\end{listing}

\subsection*{Merkhilfe für die Lambda-Syntax}
\begin{align*}
	(\text{LambdaParameter}) \rightarrow \{\text{Anweisungen}\}
\end{align*}
\end{document}  







