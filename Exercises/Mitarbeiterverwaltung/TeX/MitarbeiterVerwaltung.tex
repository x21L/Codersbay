\documentclass[12pt, oneside]{article}   	% use "amsart" instead of "article" for AMSLaTeX format
\usepackage{geometry}                		% See geometry.pdf to learn the layout options. There are lots.
\geometry{letterpaper}                   		% ... or a4paper or a5paper or ... 
%\geometry{landscape}                		% Activate for rotated page geometry
%\usepackage[parfill]{parskip}    		% Activate to begin paragraphs with an empty line rather than an indent
\usepackage{graphicx}				% Use pdf, png, jpg, or eps§ with pdflatex; use eps in DVI mode
								% TeX will automatically convert eps --> pdf in pdflatex	
\usepackage{xcolor}
				
\usepackage{minted}
								
\usepackage{amssymb}

%SetFonts

%SetFonts


\title{
Mitarbeiterverwaltung \\
\vspace{0.3cm}
\large Einführung in die Objektorientierung
}
%\author{Lukas Wais}
\date{}							% Activate to display a given date or no date

\begin{document}
 \setlength{\parindent}{0em} 
\maketitle

\section{Aufgabe}

Erstelle eine Klasse \textit{Department}, diese Klasse repräsentiert eine Abteilung einer Firma. \\  \\
Folgende Eigenschaften muss eine Abteilung aufweisen:
\begin{itemize}
	\item name
	\item id
	\item country
	\item city
\end{itemize}
\vspace{0.5cm}

Es soll möglich sein eine "default"-Abteilung erstellen zu können, diese hat die \\ Attribute:
\begin{itemize}
	\item name = "Test"
	\item id = 123
	\item country = "Neverland"
	\item city = "Gotham"
\end{itemize}
\vspace{0.5cm}

\section{Aufgabe}

Erstelle eine Klasse \textit{Employee}, diese Klasse repräsentiert einen Mitarbeiter in einer Firma. \\  \\
Folgende Eigenschaften muss ein Mitarbeiter aufweisen:
\begin{itemize}
	\item firstName
	\item lastName
	\item id
	\item department
\end{itemize}
\vspace{0.5cm}

Des weiteren soll es möglich sein, einen "default" Mitarbeiter anlegen zu können. Dieser soll standardmäßig wie folgt definiert sein:
\begin{itemize}
	\item firstName = "Max"
	\item lastName = "Mustermann"
	\item id = 123
	\item Test Abteilung von der \textit{Department} - Klasse
\end{itemize}
\vspace{1cm}

Die Methode \textit{print()} soll folgende Konsolenausgabe, dynamisch für jeden \\ Mitarbeiter liefern: \\
\vspace{0.1cm}

Der Mitarbeiter 	123 Max Mustermanm  arbeitet in der Abteilung Test in Gotham
\vspace{1cm}

Beide Klassen \textbf{müssen} die isEquals-Methode implementieren. Diese soll auf Basis der ID die Gleichheit zweier Objekte zurückgeben.

\newpage

\section{Aufgabe}
Die Klasse \textit{App} ist die Hauptapplikation. 

\subsection{Grundfunktionen}
Implementiere eine \textit{main}-Methode, in der die Funktionen der vorher erstellten Klassen getestet werden können. \\ \\

\textbf{Bonus: } Die IDs sollen zufällig mit Hilfe einer Methode generiert werden.



\subsection{Konsolenanwendung}
Programmiere eine Konsolenanwendung, mit der beliebige Mitarbeiter und \\ Departments erzeugt werden können. \\ \\
Ein Dialog könnte wie folgt aussehen: \\ \\
\fbox{\parbox{\linewidth}{
Bitte legen Sie eine Abteilung an: \\ \\
Name \textcolor{magenta}{Development} \\
Land \textcolor{magenta}{Deutschland} \\
Stadt \textcolor{magenta}{Berlin} \\ \\
Bitte legen Sie einen Mitarbeiter an: \\ \\
Vorname \textcolor{magenta}{Hans} \\
Nachname \textcolor{magenta}{Müller} \\ \\
Sie haben folgenden Mitarbeiter angelegt:

Der Mitarbeiter 	668	Hans	Müller arbeitet in der Abteilung Development in Berlin \\ \\
Möchten Sie noch einen Mitarbeiter anlegen? \\
j für ja und n für beenden \\
\textcolor{magenta}{n} \\ \\
Auf Wiedersehen!
}}

\newpage
\subsection*{Für die Aufgabe steht folgendes Skeleton zur Verfügung:}

\begin{listing}[ht]
\inputminted[frame=lines, style=fruity, bgcolor=black]{Java}{App.java}
\caption{App.java}
\end{listing}
\newpage
\begin{listing}[ht]
\inputminted[frame=lines, style=fruity, bgcolor=black]{Java}{Employee.java}
\caption{Employee.java}
\end{listing}
\newpage
\begin{listing}[ht]
\inputminted[frame=lines, style=fruity, bgcolor=black]{Java}{Department.java}
\caption{Department.java}
\end{listing}

\newpage
\section{Zusatz}
\begin{itemize}
	\item Alle Mitarbeiter sollen in ein Array gespeichert werden.
	\item Wurde der Mitarbeiter erfolgreich angelegt, soll er ausgegeben werden.
	\item Nach beenden des Anlegevorgangs sollen alle angelegten Mitarbeiter 			ausgegeben werden.
	\item Ist das Array voll, also kann kein Mitarbeiter mehr eingefügt werden soll 			das Array verdoppelt werden.
	\item Während der Erstellung neuer Mitarbeiter soll es die Möglichkeit geben 			aus den jeweiligen Departments zu wählen.
\end{itemize}

\subsection*{Hinweise}
\begin{itemize}
	\item Achtung auf \textit{NullPointerExceptions} bei Arrays!
	\item Mit \textit{.equalsIgnoreCase("")} können Strings verglichen 	werden. Hier wird nicht auf die Groß-/Kleinschreibung geachtet beim vergleich A.equals("a").

	\item Eine foreach-Schleife könnte hilfreich sein. Ist eventuell lesbarer.

	\item Nicht auf das Schlüsselwort static vergessen.

	\item Die IDs sollen bei jedem Mitarbeiter eindeutig sein.\end{itemize}
\end{document}  







