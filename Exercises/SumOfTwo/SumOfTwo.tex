\documentclass[12pt, oneside]{article}
%packages
\usepackage{geometry}                		                  		
\usepackage{graphicx}				
\usepackage{xcolor}
\usepackage{minted}							
\usepackage{amssymb}
\PassOptionsToPackage{hyphens}{url}\usepackage{hyperref}
\usepackage[utf8]{inputenc}
\usepackage[T1]{fontenc}
\usepackage{tikzsymbols}
\usepackage[naustrian]{babel}
\usepackage{amsmath}
%layout
\geometry{letterpaper} 

\title{
Coding Challenge \\
\Large{SumOfTwo}
\vspace{0.3cm}
}
\begin{document}
 \setlength{\parindent}{0em} 
\maketitle

\section{Das Problem}
Gegeben sind zwei int-Arrays a und b, mit beliebigen Werten und ein Zielwert \\ (target value) v. Finde heraus, ob es ein Nummernpaar gibt, wobei ein Wert aus a und der andere Wert aus b sein muss, die in Summe den Zielwert v ergeben. Es soll true zurückgegeben werden, falls es so ein Paar gibt, andernfalls false.\\

\section{Aufgabe}
Schreibe die Methode \textit{sumOfTwo(int[] a, int[] b, int v)} wie oben beschrieben. \\ Die Lösung soll möglichst effizient sein, also \textcolor{red}{\textbf{kein!!}} Bruteforce.

\section{Beispiele}
Hier folgen zwei Beispiele für das oben beschriebene Problem:
\begin{minted}{Java}
int[] a = [1,2,3];
int[] b = [10,20,30,40];
int v = 42;
\end{minted}

Die Methode \textit{sumOfTwo} gibt true zurück, da $40+2=42$ ist.
\newpage
a und b müssen nicht sortiert sein und können auch mehrmals den gleichen Wert beinhalten. 
\begin{minted}{Java}
int[] a = {0,0,-5,30212};
int[] b = {-10,40,-3,9};
int v = -8;
\end{minted}
Es wird wieder true returned, da $-5 + (-3) = -8$ ist.

\section{Hinweise}
\begin{itemize}
	\item \textcolor{red}{Wichtig: } Denke daran, dass Arrays in Java $2^{31}-1$ Elemente enthalten können $\rightarrow$ kein Bruteforce.
	\item HashSets sind hilfreich, da sie einfügen und lesen in effizienter und vor allem konstanter Zeit ermöglichen. \\
		\url{https://docs.oracle.com/en/java/javase/12/docs/api/java.base/java/util/HashSet.html} \\
	\item Natürlich könnt ihr die Aufgabe auch in Kotlin schreiben \Winkey
	\end{itemize}
	
	\begin{center}
	\begin{minted}[frame=single]{Java}
	// HashSet für int - Werte
	HashSet<Integer> hashSet = new HashSet<>();
	\end{minted}
	\end{center}
Mit Bruteforce ist gemeint, dass jedes einzelne Element aus a mit jedem einzelnen Element aus b addiert wird und danach mit v verglichen wird. Hier ein Beispiel:
\begin{center}
\begin{align*}
	a = [1,2,3] \\
	b = [4,5,6] \\
	v = 7 \\ \\
	a[0] = 1 \\
	b[0] + a[0] = 7? \\
	. \\
	. \\
	b[3] + a[0] = 7 \\
	\text{return true}
	\end{align*}
\end{center}
\end{document}  







