\documentclass{beamer}

\usepackage[utf8]{inputenc}
% \usepackage{beamerthemesplit} // Activate for custom appearance

% Theme
\usetheme{Madrid}  %% Themenwahl

% pictures
\usepackage{graphicx}

%listings
\usepackage{listings}
\usepackage{color}

\definecolor{mygreen}{rgb}{0,0.6,0}
\definecolor{mygray}{rgb}{0.5,0.5,0.5}
\definecolor{mymauve}{rgb}{0.58,0,0.82}

\lstset{ 
  backgroundcolor=\color{white},   % choose the background color; you must add \usepackage{color} or \usepackage{xcolor}; should come as last argument
  basicstyle=\footnotesize,        % the size of the fonts that are used for the code
  breakatwhitespace=false,         % sets if automatic breaks should only happen at whitespace
  breaklines=true,                 % sets automatic line breaking
  captionpos=b,                    % sets the caption-position to bottom
  commentstyle=\color{mygreen},    % comment style
  deletekeywords={...},            % if you want to delete keywords from the given language
  escapeinside={\%*}{*)},          % if you want to add LaTeX within your code
  extendedchars=true,              % lets you use non-ASCII characters; for 8-bits encodings only, does not work with UTF-8
  firstnumber=1,                % start line enumeration with line 1000
  frame=single,	                   % adds a frame around the code
  keepspaces=true,                 % keeps spaces in text, useful for keeping indentation of code (possibly needs columns=flexible)
  keywordstyle=\color{blue},       % keyword style
  language=Java,                 % the language of the code
  morekeywords={*,...},            % if you want to add more keywords to the set
  numbers=none,                    % where to put the line-numbers; possible values are (none, left, right)
  numbersep=5pt,                   % how far the line-numbers are from the code
  numberstyle=\tiny\color{mygray}, % the style that is used for the line-numbers
  rulecolor=\color{black},         % if not set, the frame-color may be changed on line-breaks within not-black text (e.g. comments (green here))
  showspaces=false,                % show spaces everywhere adding particular underscores; it overrides 'showstringspaces'
  showstringspaces=false,          % underline spaces within strings only
  showtabs=false,                  % show tabs within strings adding particular underscores
  stepnumber=2,                    % the step between two line-numbers. If it's 1, each line will be numbered
  stringstyle=\color{mymauve},     % string literal style
  tabsize=2,	                   % sets default tabsize to 2 spaces
  title= title               % show the filename of files included with \lstinputlisting; also try caption instead of title 
}

\title{Welcome to the CODERS.BAY}
\author{Lukas Wais}

\date{\today}

\begin{document}

\frame{\titlepage}

\section[Outline]{}
\frame{\tableofcontents}

\section{Introduction}
\subsection{Whoami?}
\frame {
  \frametitle{Lukas Wais}

  \begin{itemize}
  % \item<1> Normal LaTeX class.
  % \item<2-> Easy overlays.
  % \item<3-> No external programs needed.      
    \item HTBLA Traun für Informations~- und Kommunikationstechnik.
    \item Webdeveloper und System Administrator 4YOUcard.
    \item Informatikstudium JKU Linz.
  \end{itemize}
}

\subsection{Do we really need code?}
\frame {
  \frametitle {Code is everywhere}
 	\includegraphics[width=\linewidth]{tablet.png}
}

\frame {
  \frametitle {Computers on Wheels}
 	\includegraphics[width=\linewidth]{tesla}
}

\section{Coding guidelines}
 
  \frame {
    \frametitle {Language}
    \begin{center}
    	\includegraphics[width=0.5\linewidth]{england}
	\vspace{0.5cm}
    	 \begin{itemize}   
   		\item variables, comments and classes MUST be named in English 
  	\end{itemize}
  	\end{center}
  }
  
  \subsection{Coding Style}
\frame {
  \frametitle {Coding Style}
  \lstinputlisting[title= Egyptian Brackets ]{do.java}
  \lstinputlisting[title= Do NOT do this]{dont.java}
  }
  
  \frame {
   \frametitle {Comments and Style}
    \lstinputlisting[caption = ]{comments.java}
    
    \begin{itemize}   
   	\item use tabs not whitespaces.
    	\item Camel Casing.
    	\item Classes and Types Pascal Casing.
	\item Constants upper case.
	\item do not use underscores  TEMP\textunderscore KELVIN $\rightarrow$ old style
  \end{itemize}
  }
  
  \frame {
   \begin {block}{Martin Fowler said:}
    "Any fool can write code that a computer can understand. \\
    Good programmers write code that humans can understand."
  \end {block}
}

\section {Whats is JAVA?}
\subsection {This is Java}
\frame {
  \frametitle {This is JAVA}
 	\includegraphics[width=\linewidth]{java}
}

 \frame {
    \frametitle {This is also JAVA}
    \lstinputlisting[caption = ,  firstline=24, lastline=41]{example.java}
  }

\subsection {JDK and JRE}
\frame {
 \frametitle {Java Versions}
 
     \begin{itemize}   
   	\item JDK \ldots Java Development Kit $\rightarrow$ for developing JAVA applications.
    	\item JRE \ldots Java Runtime Environment $\rightarrow$ for running JAVA applications.
	\item JDK Versions $>$ 8. 
  \end{itemize}
  
}

\subsection {Compiling}
\frame {
\frametitle {The Java Compilation Process}

\begin{center}
    	\includegraphics[width=\linewidth]{JVM}
\end{center}
}

\frame{}

\end{document}








