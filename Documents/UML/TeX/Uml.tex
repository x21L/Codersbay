\documentclass[12pt, oneside]{article}   	% use "amsart" instead of "article" for AMSLaTeX format
\usepackage{geometry}                		% See geometry.pdf to learn the layout options. There are lots.
\geometry{a4paper}                   		% ... or a4paper or a5paper or ... 
%\geometry{landscape}                		% Activate for rotated page geometry
%\usepackage[parfill]{parskip}    		% Activate to begin paragraphs with an empty line rather than an indent
\usepackage{graphicx}				% Use pdf, png, jpg, or eps§ with pdflatex; use eps in DVI mode
						% TeX will automatically convert eps --> pdf in pdflatex		
\usepackage{amssymb}
\usepackage{tikz}
\usepackage{pgf-umlcd}
%SetFonts
\usepackage{url}
\usepackage{hyperref}
\usepackage[naustrian]{babel}
\setlength\parindent{0pt}
\usepackage{listings}
\lstset{language=Java}
\title{UML Klassendiagramme}
\author{}
\date{}							% Activate to display a given date or no date

\begin{document}
\maketitle
\tableofcontents

\newpage

\section{Notation}
%\subsection{}
Eine regul\"are Klasse wird wie folgt dargestellt: \\
Oben stehen die Klassenvariablen und im unteren Teil werden die Methoden eingezeichnet. Es ist auch m\"oglich anstatt der hier verwendeten Pascalnotation die regul\"are Java Syntax zu verwenden, wie etwa f\"ur Variablen \textit{String name} oder f\"ur Methoden \\ \textit{void calcHeight(int x, int y).}\\
\vspace{0.5 cm}
\begin{center}
\begin{tikzpicture}
  \begin{class}[text width=9cm]{ClassName}{0,0}
    \attribute{name : attribute type}
    \attribute{name : attribute type = default value}

    \operation{name(parameter list) : type of value returned}
    % virtual operation
    \operation[0]{name(parameters list) : type of value returned}
  \end{class}
\end{tikzpicture}
\end{center}

%\vspace{0.5cm}

\subsection{Sichtbarkeit}
Im Regelfall werden nur die \"offentlichen Schnittstellen eingezeichnet, aufgrund der \"Ubersicht. M\"ochte man jedoch mehr ins Detail gehen, gibt es folgende Zeichen um die Sichtbarkeit zu kennzeichnen.
\begin{enumerate}
\item $+ \ldots$ public
\item $\# \ldots$ protected
\item entweder keine Kennzeichnung oder $\sim \ldots$ package private
\item $- \ldots$ privat
\end{enumerate}

%\vspace{0.5cm}

\begin{center}
	\begin{tikzpicture}%[show background grid]
		\begin{class}[text width=8cm]{BankAccount}{0,0}
			\attribute{+ owner : String}
			\attribute{+ balance : Dollars}
			\operation{- deposit( amount : Dollars )}
			\operation{$\sim$ withdrawal( amount : Dollars )} \operation{\#
			updateBalance( newBalance : Dollars)} 
		\end{class}
	\end{tikzpicture}
\end{center}

\subsection{Abstrakte Klassen und Interfaces}

Für die Kennzeichnung von Interfaces und abstrakten Klassen wird $<<$interface$>>$ verwendet. Bei abstrakten Klassen kann anstatt $<<$abstract$>>$ einfach der \\ Klassenname \textit{kursiv} geschrieben werden.

\vspace{0.5cm}

\begin{center}
\begin{tikzpicture}
	%abstract class
	\begin{abstractclass}[text width=6cm]{BankAccount}{0,4}
		\attribute{owner : String}
		\attribute{balance : Dollars = 0}
		\operation{deposit(amount : Dollars)} \operation[0]{withdrawl(amount : Dollars)}
	\end{abstractclass}
	%interface
	\begin{interface}{Person}{7,4}
		\attribute{firstName : String} \attribute{lastName : String}
	\end{interface} 
	%class
	\begin{class}[text width=6cm]{$BankAccount$}{0,0}
		\attribute{owner : String}
		\attribute{balance : Dollars = 0}
		\operation{deposit(amount : Dollars)} \operation[0]{withdrawl(amount : Dollars)}
	\end{class}
\end{tikzpicture}
\end{center}
\vspace{0.5cm}

\newpage

\subsection{Vererbung}
Eine Vererbungsbeziehung wird mittels Pfeilen mit einer durchzogenen Linie dargestellt.

\begin{center}
\begin{tikzpicture}[scale=0.8]
	%BankAccount
	\begin{class}[text width=6cm]{BankAccount}{0,0} \attribute{owner : String}
		\attribute{balance : Dollars = 0}
		\operation{deposit(amount : Dollars)} \operation[0]{withdrawl(amount : Dollars)}
	\end{class}
	%CheckingAccount
	\begin{class}[text width=8cm]{CheckingAccount }{-5,-5}
		\inherit{BankAccount} \attribute{insufficientFundsFee : Dollars}
		\operation{processCheck ( checkToProcess : Check)}
		\operation{withdrawal ( amount : Dollars )}
	\end{class}
	%SavingsAccount
	\begin{class}[text width=7cm]{SavingsAccount}{5,-5}
		\inherit{BankAccount} \attribute{annualInteresRate : Percentage}
		\operation{depositMonthlyInterest ( )}
		\operation{withdrawal ( amount : Dollars )}
	\end{class}
\end{tikzpicture}
\end{center}

Implementierungen von \textbf{Interfaces} werden mit strichlierten Pfeilen dargestellt.
\begin{center}
\begin{tikzpicture}%[show background grid]
	%Person
	\begin{interface}{Person}{0,0} \attribute{firstName : String}
		\attribute{lastName : String} \end{interface}
	%Professor
	\begin{class}{Professor}{-5,-5}
		\implement{Person}
		\attribute{salary : Dollars}
	\end{class}
	%Student
	\begin{class}{Student}{5,-5} \implement{Person}
		\attribute{major : String} \end{class}
\end{tikzpicture}
\end{center}

\newpage
\subsection{Beziehungen zwischen Klassen}

\textbf{Assozation} benutzt-Beziehung \\ 
\begin{center}
	A benutzt B
\end{center}
\begin{center}
\begin{tikzpicture}
% \draw[help lines] (-7,-6) grid (6,0);
	\begin{class}[text width=2cm]{A}{0,0}\end{class}
	\begin{class}[text width=2cm]{B}{9,0}\end{class}
\unidirectionalAssociation{A}{}{}{B}
\end{tikzpicture}
\end{center}

\textbf{Aggregation} hat-Beziehung \\ 
\begin{center}
	A hat ein B
\end{center}
\begin{center}
\begin{tikzpicture}
% \draw[help lines] (-7,-6) grid (6,0);
	\begin{class}[text width=2cm]{A}{0,0}\end{class}
	\begin{class}[text width=2cm]{B}{9,0}\end{class}
\aggregation{A}{}{}{B}
\end{tikzpicture}
\end{center}

\textbf{Generalisierung} ist-Beziehung \\ 
\begin{center}
	B ist ein A
\end{center}
\begin{center}
\begin{tikzpicture}
% \draw[help lines] (-7,-6) grid (6,0);
	\begin{class}[text width=2cm]{A}{0,0}\end{class}
	\begin{class}[text width=2cm]{B}{0,-2}\inherit{A}\end{class}
\end{tikzpicture}
\end{center}

\vspace{0.5cm}

\textbf{hat-Beziehung:} wenn man sagen kann:  \glqq besteht aus\grqq. A hat permanente Referenz auf B.

\begin{lstlisting}[frame=single]
class A {
	B b;
	...
}
\end{lstlisting}
\begin{center}
\begin{tikzpicture}
% \draw[help lines] (-7,-6) grid (6,0);
	\begin{class}[text width=2cm]{A}{0,0}\end{class}
	\begin{class}[text width=2cm]{B}{9,0}\end{class}
\aggregation{A}{b}{}{B}
\end{tikzpicture}
\end{center}

\newpage

\textbf{benutzt-Beziehung:} wenn man sagen kann:  \glqq steht in Verbindung mit\grqq. A hat nur tempor\"are Beziehung zu B und C.

\begin{lstlisting}[frame=single]
class A {
	void foo(B b) {
		b.bar();
		C c = ...;
	}
	...
}
\end{lstlisting}

\begin{center}
\begin{tikzpicture}
% \draw[help lines] (-7,-6) grid (6,0);
	\begin{class}[text width=2cm]{A}{0,0}\end{class}
	\begin{class}[text width=2cm]{B}{9,0}\end{class}
	\begin{class}[text width=2cm]{C}{9,-2}\end{class}
	\unidirectionalAssociation{A}{}{}{B}
	\unidirectionalAssociation{A}{}{}{C}
\end{tikzpicture}
\end{center}

\vspace{1cm}
Oft wird zwischen beiden Beziehungen nicht unterschieden. Beides wird meist als benutzt-Beziehung modelliert.

\subsection{Kardinalit\"aten von Beziehungen}

\begin{center}
Jedes A benutzt genau $1$ B \\ \vspace{0.5cm}
\begin{tikzpicture}
	\begin{class}[text width=2cm]{A}{0,0} \end{class}
	\begin{class}[text width=2cm]{B}{5,0}\end{class}
	\unidirectionalAssociation{A}{}{}{B}
\end{tikzpicture}
\end{center}

\begin{center}
Jedes A benutzt genau $3$ B und jedes B wird von 2 A benutzt \\ \vspace{0.5cm}
\begin{tikzpicture}
	\begin{class}[text width=2cm]{A}{0,0} \end{class}
	\begin{class}[text width=2cm]{B}{5,0}\end{class}
	\association{A}{B}{2}{B}{3}{A}
\end{tikzpicture}
\end{center}

\begin{center}
Jedes A benutzt genau $2$ bis $4$ B \\ \vspace{0.5cm}
\begin{tikzpicture}
	\begin{class}[text width=2cm]{A}{0,0} \end{class}
	\begin{class}[text width=2cm]{B}{5,0}\end{class}
	\unidirectionalAssociation{A}{}{2..4}{B}
\end{tikzpicture}
\end{center}

\begin{center}
Jedes A benutzt kann ein B oder nicht (Optinialit\"at) \\ \vspace{0.5cm}
\begin{tikzpicture}
	\begin{class}[text width=2cm]{A}{0,0} \end{class}
	\begin{class}[text width=2cm]{B}{5,0}\end{class}
	\unidirectionalAssociation{A}{}{0..1}{B}
\end{tikzpicture}
\end{center}

\begin{center}
Jedes A kann null oder beliebig viele B benutzen \\ \vspace{0.5cm}
\begin{tikzpicture}
	\begin{class}[text width=2cm]{A}{0,0} \end{class}
	\begin{class}[text width=2cm]{B}{5,0}\end{class}
	\unidirectionalAssociation{A}{}{*}{B}
\end{tikzpicture}
\end{center}

Die Kardinalit\"at wird am Anfang des Modellierens oft weggelassen, au\ss{}er *.

\subsection{Packages}
Nat\"urlich lassen sich auch Packages darstellen. \textit{Accounts} ist der Name des Packages.

\begin{center}
\begin{tikzpicture} [scale=0.70]
	%Accounts
	\begin{package}{Accounts}
	%BankAccount
	\begin{class}[text width=7cm]{BankAccount}{0,0} 
		\attribute{owner : String}
		\attribute{balance : Dollars = 0}
		\operation{deposit(amount : Dollars)} 
		\operation[0]{withdrawl(amount : Dollars)}
	\end{class}
	%CheckingAccount
	\begin{class}[text width=6cm]{CheckingAccount }{-5,-5}
		\inherit{BankAccount} \attribute{insufficientFundsFee : Dollars}
		\operation{withdrawal (amount : Dollars)}
	\end{class}
	%SavingsAccount
	\begin{class}[text width=6cm]{SavingsAccount}{5,-5}
		\inherit{BankAccount} 
		\attribute{annualInteresRate : Percentage}
		\operation{withdrawal (amount : Dollars)}
	\end{class}
	\end{package}
\end{tikzpicture}
\end{center}

\newpage

\section{Beispiel}
\subsection{Webshop}
\begin{center}
\begin{tikzpicture}[scale=0.7, show background grid]
	%Shopping
	\begin{interface}[text width=4cm]{Shopping}{0,0} 
		\operation{addItem()} 
		\operation{removeItem()}
		\operation{placeOrder()}
		\operation{cancelOrder()}
	\end{interface}
	%ShoppingCart
	\begin{class}[text width=4cm]{ShoppingCart}{0,-8}
		\inherit{Shopping} 
		\attribute{total : int}
		\attribute{tax : int}
		\operation{addItem()} 
		\operation{removeItem()}
		\operation{placeOrder()}
		\operation{cancelOrder()}
	\end{class}
	%Customer
	\begin{class}[text width=4cm]{Customer}{18,-9}
		\attribute{name : String}
		\attribute{address : String}
		\attribute{email : String}
	\end{class}
	%SpecialCustomer
	\begin{class}[text width=4cm]{SpecialCustomer}{18,-15.2}
		\attribute{discount : int}
		\operation{selectCard()}
		\inherit{Customer}
	\end{class}
	%CreditCard
	\begin{class}[text width=4cm]{CreditCard}{8,-14.5}
		\attribute{owner : String}
		\attribute{number : long}
		\attribute{expires : Date}
		\operation{charge()}
	\end{class}
	%Product
	\begin{class}[text width=4cm]{Product}{12,-19.7}
		\attribute{name : String}
		\attribute{price : int}
	\end{class}
	%Item and relations
	\begin{class}[text width=4cm]{Item}{0,-20}
		\attribute{quantity : int}
	\end{class}
	%Associations
	\unidirectionalAssociation{ShoppingCart}{card}{0..1}{CreditCard}
	\unidirectionalAssociation{SpecialCustomer}{cards}{*}{CreditCard}
	\unidirectionalAssociation{ShoppingCart}{customer}{}{Customer}
	\unidirectionalAssociation{Customer}{}{cart}{ShoppingCart}
	\unidirectionalAssociation{ShoppingCart}{items}{*}{Item}
	\unidirectionalAssociation{Item}{product}{*}{Product}
\end{tikzpicture}
\end{center}

\section{Weiterf\"uhrende Links}
\begin{itemize}
	\item \url{https://www.uml-diagrams.org/}
	\item \url{https://www.omg.org/spec/UML/2.5.1/PDF}
	\item \url{http://uml.org/}
	\item \url{http://uml.org/resource-hub.htm}
	\item \url{https://www.eclipse.org/papyrus/}
\end{itemize}
\end{document}



