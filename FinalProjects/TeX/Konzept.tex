\documentclass[12pt, oneside]{article}
%packages
\usepackage{geometry}                		                  		
\usepackage{graphicx}				
\usepackage{xcolor}
\usepackage{graphicx}
						
\usepackage{amssymb}
\usepackage[utf8]{inputenc}
\usepackage[T1]{fontenc}
\usepackage[naustrian]{babel}
\usepackage{tcolorbox}
%layout
\geometry{letterpaper} 

\title{Konzept}
\begin{document}
 \setlength{\parindent}{0em} 
\maketitle

\section{Die Aufgabe}
Es soll ein Konzept, mit einem Textverarbeitungsprogramm eurer Wahl \\(\LaTeX, Markdown, \ldots), angefertigt werden. Dieses Dokument ist primär nicht an Techniker gerichtet, sondern an Recruter, oder Business Angels. Ihr sollt euer \\ Produkt verkaufen.

\subsection*{Struktur}
\begin{itemize}
	\item Was will ich entwickeln?
	\item Warum will ich es entwickeln, welche Use Cases gibt es dafür? Habe ich einen Unique Selling Point gegenüber anderen Produkten?
	\item Wie setzte ich das Projekt um, mit welche Werkzeugen?
	\item Unter welcher Lizenz veröffentliche ich mein Projekt?\\ Open Source vs. Closed Source.
	\item Falls es eine Benutzeroberfläche gibt: erste Mock Ups davon.
\end{itemize}

Wie bereits besprochen werden wir das Projekt digital über Discord besprechen. \\ Erster vorläufiger Termin für das Konzept: 26. März 2020. 
\\ \\Etwa eine Woche darauf wird das Gantt-Diagram fällig sein. Dazu folgen noch ein paar Informationen.
\\
\begin{tcolorbox}
	Nützt die freie Zeit um einige Tools/Frameworks/etc. zu testen. Vielleicht auch das ein oder andere Build Tool wie Maven. Für Maven gibt es Folien in Moodle.
\end{tcolorbox}

\end{document}  







