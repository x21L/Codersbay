\documentclass[12pt, oneside]{article}
%packages
\usepackage{geometry}                		                  		
\usepackage{graphicx}				
\usepackage{xcolor}							
\usepackage{amssymb}
\PassOptionsToPackage{hyphens}{url}\usepackage{hyperref}
\usepackage[utf8]{inputenc}
\usepackage[T1]{fontenc}
\usepackage[naustrian]{babel}
\usepackage{amsmath}
\usepackage{tcolorbox}
%layout
\geometry{letterpaper} 

\title{Abschlussprojekt}
\begin{document}
 \setlength{\parindent}{0em} 
\maketitle

\section{Folgende technische Anforderungen muss das \\ Projekt aufweisen}
\vspace{0.5cm}
\begin{tcolorbox}
	\begin{center}
		Das Ziel ist zum Abschluss eine lauffähige, standalone jar-Datei zu haben.
	\end{center}
\end{tcolorbox}

\begin{itemize}
	\item Maximale Abstraktion mit interfaces, abstract, \ldots.
	\item Clean Code, Wiederverwendbarkeit, lesbarkeit $-$ 	\glqq do not repeat yourself	\grqq.
	\item Datenkapselung, wenn es sinnvoll ist (privat, final).
	\item Verwendung von vertiefenden Inhalten wie: abstrakte Datenstrukturen, Enums, Generics, Lambdas, \ldots
	\item Verwendung eines Version Controls Systems, wie Github.
	\item Saubere JDBC-Schnittstelle.
	\item Um die Funktionsweise sicher zu stellen, soll das Projekt mit JUnit-Tests \\ getestet werden.
\end{itemize}

\subsection*{Folgende zwei Möglichkeiten stehen zur Auswahl}
\begin{itemize}
	\item Implementierung einer grafischen Benutzeroberfläche, GUI.
\end{itemize}
\begin{center}
	oder
\end{center}
\begin{itemize}
	\item Verwendung eines Design Patterns, ausgenommen Singleton.
\end{itemize}

\section{Nicht technische Anforderungen}
Was wäre ein erfolgreiches Projekt ohne etwas Projektmanagement und \\ Dokumentation?

\begin{itemize}
	\item Konzept
	\item Gantt-Diagramm
	\item UML-Diagramm
	\item Javadoc
	\item Zum Abschluss des Projekts eine Dokumentation/Handbuch zur Verwendung der Software.
	\item Es wird alle ein bis zwei Wochen eine Projektpräsentation geben. Um den anderen den Fortschritt und neu gelerntes zu präsentieren. Damit alle verschiedene Themengebiete und Technologien kennenlernen.
\end{itemize}
\end{document}  







